\newcommand{\abs}[1]{
  \ensuremath
  \lvert #1 \rvert
}
\newcommand{\norm}[1]{
  \ensuremath
  \lVert #1 \rVert
}
\newcommand*{\defeq}{
  \ensuremath
  \stackrel{\text{def}}{=}
}
\newcommand*{\zzeq}{
  \ensuremath
  \stackrel{\text{z.z.}}{=}
}
\newcommand*{\defiff}{
  \ensuremath
  \stackrel{\text{def}}{\iff}
}
\newcommand*{\union}{
  \ensuremath
  \: \cup \:
}
\newcommand*{\inter}{
  \ensuremath
  \: \cap \:
}
\newcommand{\uproman}[1]{\uppercase\expandafter{\romannumeral#1}}
\newcommand{\lowroman}[1]{\romannumeral#1\relax}
\newcommand{\limes}[1]{
  \ensuremath
  \displaystyle{\lim_{#1}}
}
\newcommand{\Ker}{
  \ensuremath
  \operatorname{Ker}
}
\newcommand{\Img}{
  \ensuremath
  \operatorname{Im}
}
\newcommand{\Hom}{
  \ensuremath
  \operatorname{Hom}
}
\newcommand{\inv}{
  \ensuremath
  \operatorname{inv}
}
\newcommand{\sgn}{
  \ensuremath
  \operatorname{sgn}
}


\newcounter{definitioncounter}
\newtcolorbox[auto counter, number within=chapter]{define}[2][]{
  colback=blue!5!white,
  colframe=blue!75!black,
  fonttitle=\bfseries,
  title={Definition: #2},
  sharp corners=south,
  enhanced,
  #1
}

\newcounter{propositioncounter}
\newtcolorbox[auto counter, number within=chapter]{proposition}[2][]{
  colback=green!5!white,
  colframe=green!60!black,
  fonttitle=\bfseries,
  title=Proposition: {#2},
  sharp corners=south,
  enhanced,
  #1
}

\newcounter{corollarycounter}
\newtcolorbox[auto counter, number within=chapter]{corollary}[2][]{
  colback=purple!5!white,
  colframe=purple!60!black,
  fonttitle=\bfseries,
  title=Korollar: #2,
  sharp corners=south,
  enhanced,
  #1
}

\newcounter{lemmacounter}
\definecolor{turquoise}{RGB}{64, 224, 208}
\newtcolorbox[auto counter, number within=chapter]{lemma}[2][]{
  colback=turquoise!5!white,
  colframe=turquoise!60!black,
  fonttitle=\bfseries,
  title=Lemma: #2,
  sharp corners=south,
  enhanced,
  #1
}

\newcounter{anwendungscounter}
\newtcolorbox[auto counter, number within=chapter]{anwendung}[2][]{
  colback=blue!5!white,
  colframe=blue!75!black,
  fonttitle=\bfseries,
  title={Anwendung: #2},
  sharp corners=south,
  enhanced,
  #1
}

\NewDocumentCommand{\example}{ o }{
  \noindent\textit{Beispiel
    \IfValueTF{#1}{ (#1).}{.}
  }
}

\NewDocumentCommand{\anmerkung}{ o }{
  \noindent\textit{Anmerkung
    \IfValueTF{#1}{ (#1).}{.}
  }
}